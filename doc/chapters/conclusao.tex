\chapter{Conclusão}
O novo ramo da Inteligência Artificial, \textit{General Game Playing}, ainda não conta com técnicas consolidadas para a resolução dos problemas propostos. O que torna uma área excelente para desenvolvimento de novos métodos, como extração de características comuns em jogos, ou aplicação de métodos usados comumente em outras áreas, como Monte Carlo.

Neste trabalho foi apresentado a construção de um jogador GGP utilizando as técnicas de Monte Carlo e UCT. Partindo da linguagem utilizada na descrição dos jogos, a mecânica de como derivar as informações também foi exposta. 

No Capítulo 5 foi mostrado que o método de Monte Carlo por si só não é suficiente para a construção do jogador. O aprimoramento com Upper Confidence Bonus aumenta a taxa de sucesso enormemente. 

Os detalhes da implementação de um jogador, utilizando os conceitos apresentados nos capítulos anteriores, foram mostrados no Capítulo seguinte. Peculiaridades da linguagem GDL devem ser levadas em consideração na transcrição para Prolog. A comunicação entre o provador de teoremas e o jogador é um fator muito importante no desempenho. Já a implementação do algoritmo UCT é praticamente direta, variando-se apenas a constante de utilização do bônus.

No entanto, como exposto no resultados, ainda há muito o que progredir. Na implementação alternativas e melhoramentos na seleção das ações devem ser incorporados ao jogador para aumentar a taxa de sucesso.