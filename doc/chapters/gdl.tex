\chapter{Game Description Language}
\label{gdl}

Para que a construção de um sistema autônomo em GGP seja possível, é preciso uma linguagem padrão para a descrição dos jogos. \textit{Game Description Language}\cite{gdl}(GDL), derivada da lógica de Primeira Ordem e um subconjunto de \textit{Knowledge Interchage Format}, é a linguagem usada em todas as competições realizadas até hoje. GDL é puramente axiomática, nenhuma algebra está inclusa linguagem. Caso o jogo necessite, as regras aritméticas pertinentes devem ser incluídas na descrição.

A classe de jogos que pode ser descrita em GDL pode ser classificada como \textit{n-jogadores} ($n \geq 1 $), \textit{determinística}, \textit{perfeitamente informada} com \textit{movimentos simultâneos}. "Determinística", exclui os jogos que contenham elementos aleatórios, como Gamão, onde a movimentação das peças é feita de acordo com um lançamento de dados. Já "perfeitamente informado" proíbe que qualquer informação sobre o estado atual seja escondida de algum jogador, o que é comum na maioria dos jogos de cartas. Finalmente "movimentos simultâneos" permite descrever jogos como Jan-ken-pon (pedra-papel-tesoura), onde todos os jogadores agem de uma vez, mas sem descartar jogos com alternância nas movimentações entre jogadores, como xadrez ou damas, restringindo todos os jogadores, exceto um, a executar apenas a ação "no-op". GDL também é \textit{finita} em muitos aspectos: o espaço de estados consiste em muitos estados finitos; há também um número finito e fixo de jogadores; cada jogador tem um número finito de possibilidades de ações em cada estado e os jogos devem ser formulados de maneira que culmine em um estado terminal com um número finito de movimentos. Cada estado terminal está associado com um valor (\textit{goal}) para cada jogador. 

Os jogos em GDL são modelados como máquinas de estado, em que cada estado pode ser descrito como um conjunto de fatos verdadeiros em um determinado tempo. Um destes estados é projetado para ser o estado inicial. As transições são determinadas combinando as ações de todos os jogadores. O jogo prossegue até que um dos estados finais seja atingido.

\section{Estrutura da linguagem}
Um pequeno conjunto de palavras chaves é usado para a descrição dos jogos: \textit{role}, \textit{init}, \textit{next}, \textit{true}, \textit{does}, \textit{terminal}, \textit{goal} e \textit{distintic}. Para ilustrar o uso na linguagem na descrição de um jogo, tomaremos como exemplo a descrição parcial do \textbf{jogo da velha}, onde os estados correspondem a configuração de preenchimento das células de uma matriz $3x3$ com \textit{x} ou \textit{o}. A descrição completa se encontra no apêndice.

\subsection{Jogo da velha em GDL}
\subsubsection*{Role}
A palavra chave \textit{role} é usada para descrever quais serão os jogadores da partida. Neste caso temos um jogador para marcar \textit{x} e outro \textit{o}. Vale notar a notação infixada que o GDL usa, diferente do Prolog que usa notação prefixada, onde a mesma declaração do jogador \textit{x} seria feita como \texttt{role(xplayer)}.
\begin{verbatim}                                                                                         
  (role xplayer)                                                                                         
  (role oplayer)                                                                                         
\end{verbatim}

\subsubsection*{Init}
O estado inicial é constituído do conjunto de argumentos dos predicados \textit{init}. O jogo da velha se inicia com todas as células em branco e o jogador \textit{x} com o direito a jogada.
\begin{verbatim}                                                                                         
  (init (cell 1 1 b))                                                                                    
  (init (cell 1 2 b))                                                                                    
\end{verbatim}
	\hspace{1.5cm} {\LARGE{ \vdots }}
\begin{verbatim}	
  (init (cell 3 3 b))                                                                                    
  (init (control xplayer))                                                                               
\end{verbatim}

\subsubsection*{True}
O predicado \textit{init} pode ser visto como um restrição da relação \textit{true}, usada para escrever o que é verdade em um estado qualquer durante o decorrer do jogo. Em contrapartida, \textit{init} só pode ser usado para a descrição do estado inicial. Após o primeiro movimento, a descrição do estado atual seria:
\begin{verbatim}                                                                                         
   (true (cell 1 1 b))
\end{verbatim}
	\hspace{1.5cm} {\LARGE{ \vdots }}
\begin{verbatim}	
   (true (cell 2 2 x))                                                                                    
   (true (control oplayer))                                                                               
\end{verbatim}
indicando que o jogador \textit{x} marcou o centro da matriz e que agora o controle passa para o jogador \textit{o}.

\subsubsection*{Next}
Análoga à palavra chave \textit{true}, o predicado \textit{next} determina quais fatos serão verdades no próximo estado. A marcação do direito à jogada é feito da seguinte maneira:
\begin{verbatim}                                                                                         
  (<= (next (control xplayer))                                                                           
      (true (control oplayer)))                                                                          
  (<= (next (control oplayer))                                                                           
      (true (control xplayer)))                                                                          
\end{verbatim}
O primeiro predicado \textit{next} mostra que na próxima rodada o controle será do jogador \textit{x} somente se no estado atual ele pertencer a seu oponente. A segunda declaração empreende o mesmo papel para indicar ao jogador \textit{o} quando será sua vez.

\subsubsection*{Legal}
As regras do jogo da velha permitem que um jogador marque uma célula apenas se ela ainda não foi marcada (contém um \textit{b}) e seja sua vez de jogar. Para descrever estas regras, a relação \textit{legal}, juntamente com os demais predicados vistos, toma como parâmetro um jogador e ação a que ele tem direito. Em GDL, termos não unificados são marcados com o símbolo de interrogação.
\begin{verbatim}                                                                                         
  (<= (legal ?player (mark ?x ?y))                                                                       
      (true (cell ?x ?y b))                                                                              
      (true (control ?player)))       
  (<= (legal xplayer noop)                                                                               
      (true (control oplayer)))                                                                          
  (<= (legal oplayer noop)                                                                               
      (true (control xplayer)))                                                                          
\end{verbatim}
A primeira relação descreve a regra fundamental do jogo da velha. As outras duas relações simulam a alternância entre o controle das jogadas, causando um dos jogadores a tomar a ação \textit{noop}\footnote{No Operation, \textit{nenhuma ação}} quando não estiver no controle.

\subsubsection*{Distinct}
Usado para distinguir duas proposições, o predicado \textit{distinct} será verdadeiro somente se seus argumentos não forem iguais.
\begin{verbatim}                                                                                         
  (<= (next (cell ?x ?y b))                                                                              
      (does ?player (mark ?m ?n))                                                                        
      (true (cell ?x ?y b))                                                                              
      (distinctCell ?x ?y ?m ?n))                                                                        
  (<= (distinctCell ?x ?y ?m ?n)                                                                         
      (distinct ?x ?m))                                                                                                                                                
  (<= (distinctCell ?x ?y ?m ?n)                                                                         
      (distinct ?y ?n))                                                                                  
\end{verbatim}

\subsubsection*{Terminal}
Para determinar se o estado atingido é um estado final, os predicados \textit{terminal} devem ser consultados. 
\begin{verbatim}                                                                                         
  (<= terminal                                                                                           
      (line x))                                                                                                                                                                                
  (<= terminal                                                                                           
      (line o))                                                                                          
  (<= terminal                                                                                           
      (not open))                                                                                        
\end{verbatim}
Destas regras tem se que o jogo acaba ao marcar uma sequência de três células, para ambos os jogadores, ou quando não há mais espaços em brancos.

\subsubsection*{Goal}
Ao final do jogo, cada jogador terá uma pontuação de acordo com o seu desempenho. 
\begin{verbatim}
  (<= (goal xplayer 100)                                                                                 
      (line x))                                                                                          
  (<= (goal xplayer 50)                                                                                  
      (not (line x))                                                                                     
      (not (line o))                                                                                     
      (not open))                                                                                        
  (<= (goal xplayer 0)                                                                                   
      (line o))   
\end{verbatim}
Este recorte da descrição revela a pontuação do jogador \textit{x} caso ele vença, empate ou perca, ganhando 100, 50 ou 0, respectivamente. As mesmas regras para o jogador \textit{o} também são encontradas na descrição.

\section{Considerações}
Algumas consideraçãoes a cerca da linguagem e suas propriedades devem ser feitas.

\subsection{Predicados auxiliares}
As relações fundamentais vistas acima contam com o uso de predicados auxiliares na descrição. Usado na relação \textit{init}, o predicado \textit{(cell ?x ?y ?p)} indica qual marca a célula \textit{xy} contém. O número de regras auxiliares que podem ser usadas na descrição é indefinido. Esta descrição do jogo da velha contém outros exemplos, como: \textit{control}, \textit{distinctCell}, \textit{line}, \textit{open} e outros.

\subsection{Nenhuma dica léxica}
Um agente GGP não pode contar com a descrição do jogo para construir uma função heurística. Exceto pelas palavras chave, nenhum outro conjunto de símbolos tem algum significado para o jogador. Na verdade, durante a competição, os símbolos usados na descrição são embaralhados para prevenir que o jogador retire alguma dica léxica. Por exemplo, uma das estruturas mais básicas dos jogos é o tabuleiro. Na competição a mesma descrição do tabuleiro poderia ser algo como:
\begin{verbatim}                                                                                         
  (init (homer bart bart santa))                                                                                    
  (init (homer bart lisa santa))                                                                                    
\end{verbatim}
	\hspace{1.5cm} {\LARGE{ \vdots }}
\begin{verbatim}	
  (init (homer magie magie santa))                                                                                                                                                           
\end{verbatim}

\subsection{Problema do Quadro (The Frame Problem)}
Segundo Ginsberg \cite{mginsberg},  é um questionamento de como representamos o fato de que as coisas tendem a continuar as mesmas na ausência de informações contrárias. Em GDL, o próximo estado sempre é definido pela relação \textit{next}, logo a regra para o problema do quadro é declarar explicitamente o que será verdadeiro após as transição de estados. 
No exemplo do predicado \textit{distinct}, as regras declaram que uma célula permanecerá em branco caso nenhum jogador a tenha marcado na rodada atual. 

\subsection{Jogos Finitos}
Como mencionado anteriormente, somente jogos de tamanho finito são permitidos.  Enquanto jogos como jogo da velha são garantidamente finitos, a maioria dos jogos, como o mundo dos blocos, tem ciclos em suas máquinas de estados que permitem uma infinita sequência de ações (como por exemplo, empilhar e desempilhar o mesmo bloco indefinidamente). Portanto o construtor da descrição de garantir que o mesmo estado não seja atingido duas vezes na mesma partida, e que todas as partidas invariavelmente atinjam um estado final. A maneira mais comum para fazer isso é adicionar a descrição um contador (\textit{step}), incrementar o seu valor a cada ação e acabar o jogo quando um certo número máximo for atingido.