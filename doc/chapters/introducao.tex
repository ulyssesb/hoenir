\chapter{Introdução}
\label{intro}

General Game Playing (GGP) é uma nova área de pesquisa na Inteligência Artificial, cujo o desafio esta em construir um agente autônomo que consiga jogar efetivamente jogos que nunca viu antes. Ao contrario dos jogadores clássicos, especializados em apenas um jogo, como xadrez ou damas, as propriedades do jogo são desconhecidas pelo programador durante a construção do sistema. Isso demanda o uso e integração de várias técnicas da Inteligência Artificial, o que torna GGP ideal para o desenvolvimento de novos métodos. Desde de 2005, o \textit{Stanford Logic Group} organiza uma competição para incentivar pesquisas nesta área. Patrocinada pela \textit{American Association for Artificial Intelligence}, este evento oferece a oportunidade de comparar diferentes abordagens em um cenário competitivo.  

Para demonstrar inteligência, os jogadores devem realizar uma série de movimentos que levam a uma posição final vitoriosa. As decisões de qual ação deve ser tomada são buscadas na árvore do jogo com auxílio de uma função de avaliação. Sistemas construidos para jogos específicos usam técnicas como livros de abertura e banco de dados com posições de fim de jogo para incrementar a função de avaliação. O campeão mundial de xadrez Deep Blue \cite{dblue} tem um livro de abertura de mais de 4.000 posições e um sumário de 700.000 jogos de mestres do xadrez. 
Chinook \cite{chinook}, campeão mundial em damas, tem informações de mais de 443 bilhões de posições finais de jogo. A eficácia da função de avaliação impacta diretamente na busca por bons movimentos, permitindo que o sistema gaste mais tempo em áreas promissoras e menos em jogadas ruins.

Jogadores em GGP, no entanto, devem ser capazes de demonstrar inteligência mesmo em jogos que nunca tenha jogado. O problema central é construir uma função de avaliação, ou \textit{heurística}, que seja genérica o suficiente para funcionar em qualquer jogo que ele venha confrontar. Mesmo que um sistema tenha acesso a heurísticas para jogos específicos, ainda seria preciso determinar qual, ou quais, seriam aplicáveis para o atual problema enfrentado.

A competição, apesar de ser notariamente recente, mostra alguns caminhos para contornar o problema em gerar uma heurística. Fluxplayer usa lógica Fuzzy para determinar o grau em que a posição atual satisfaz logicamente uma condição de vitória. Outra possibilidade é a extração de características comuns em jogos, como número de peças e identificação de tabuleiros, para compor a função de avaliação; abordagem usada pelo Clune Player. Vencedor das competições de 2007 e 2008, Cadia Player usa técnicas de Monte Carlo para simular jogos e descobrir qual dos caminhos tem a maior chance de vitória.

O presente trabalho mostrará a construção de um jogador utilizando os métodos de Monte Carlo na função de avaliação, porém sem visar a competição, ficando restrito a jogos não adversaristas. O texto na seções seguintes está organizado da seguinte maneira: capítulo 2 mostra a estrutura da linguagem utilizada para a descrição dos jogos em GGP. O capitulo seguinte fará uma visão geral das técnicas clássicas da Inteligência Artificial usadas na construção deste sistema autônomo. Os detalhes da implementação são vistos no capítulo 5 e as conclusões são dadas no capítulo 6.