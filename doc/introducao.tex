\chapter{Introdução}
\label{intro}

Programar uma máquina para competir com seres humanos em jogos como xadrez, gamão e outros, foi a tempos um grande desafio para a computação e outras áreas do conhecimento. O DeepBlue{ref}, responsável pela vitória sobre o campeão mundial de xadrez Kasparov, é um exemplo de que podemos construir um software para desempenhar uma tarefa específica tão bem, ou melhor, que um ser humano. No entanto, a única função que esta super máquina consegue cumprir é jogar xadrez, enquanto Kasparov consegue realizar outras atividades comuns aos seres humanos, como preparar um café, por exemplo.
Muitas técnicas e algoritmos foram desenvolvidos na Inteligência Artificial, a ponto de derrotarmos o melhor jogador de xadrez do mundo. Porém, não caminhamos muito na construção de uma máquina capaz de realizar tarefas diversas com proficiência.

Um novo ramo da Inteligência Artificial chamado {\it General Game Playing} ({\bf GGP}), estuda a construção de um agente apto a resolver problemas diversos, necessitando apenas de uma descrição do mesmo. Mais especificamente, um jogador que demonstre inteligência em qualquer jogo cujo suas regras sejam conhecidas.
Para estimular pesquisas na área, a Universidade de Stanford realiza anualmente{ref} uma competição que premia os melhores jogadores GGP.

Este trabalho apresenta as etapas da construção de um jogador desta classe, examinando as características impostas pela competição, mas limitando a implementação a um jogador básico, ficando restrito a jogos não adversaristas (como {\it 8-puzzle}). 
